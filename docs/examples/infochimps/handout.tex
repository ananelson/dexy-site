\documentclass[a4paper]{tufte-handout}
\usepackage[pdftex]{graphicx}
\usepackage[latin1]{inputenc}
\usepackage{fancyvrb}
\usepackage{color}
<< pygments['pastie.tex'] >>

\title{Dexy for Big Data}
\author{Ana Nelson}
\date{July 13, 2011}

\begin{document}

\maketitle

\section{About}
More information about Dexy is available at \url{http://dexy.it}. You can follow the project @dexyit on twitter or the author @ananelson. Dexy is open source software to make you and your docs awesome. Check it out!
\\This project's sources are available at \url{http://bitbucket.org/ananelson/dexy-examples/src/infochimps}

\section{Python Script query-twitter-users.py}

Here is the original Python script in full:

\scriptsize
<< d['query-twitter-users.py|pyg|l'] >>
\normalsize

Here is the CSV data generated:
\scriptsize
\begin{Verbatim}
<< d['twitter-data.csv'] >>
\end{Verbatim}
\normalsize

\section{R Script}

\scriptsize
<< d['graph-twitter-data.R|fn|idio|r|pyg|l']['import-csv'] >>
\normalsize

\scriptsize
<< d['graph-twitter-data.R|fn|idio|r|pyg|l']['pairs'] >>
\normalsize

\includegraphics{dexy--pairs.pdf}

\section{Slide Source}

\scriptsize
<< d['slides.tex|pyg|l'] >>
\normalsize

\section{Dexy Config}
\scriptsize
\begin{Verbatim}
<< d['.dexy|dexy'] >>
\end{Verbatim}
\normalsize

\end{document}
