\documentclass{beamer}
\usepackage[latin1]{inputenc}

\usepackage{fancyvrb}
<< pygments['pastie.tex'] >>

\usetheme{Warsaw}
\title[Dexy for Big Data]{Refreshing Stories\\Dexy for Big Data}
\author{Ana Nelson}
\institute{dexy.it}
\begin{document}

\begin{frame}
\titlepage
\end{frame}


\begin{frame}{In the next 20 minutes...}
You will see how to:
\begin{itemize}
\item{Tell geeky stories without leaving your favourite text editor.}
\item{Bring coding, data analysis and writing into one awesome loop.}
\item{Recreate, update and maintain everything you write.}
\end{itemize}
\end{frame}

\begin{frame}{Tool for the job}
Dexy
\begin{itemize}
\item{Open Source (mostly MIT, some AGPL)}
\item{Written in Python}
\item{Command Line, Text Based}
\item{* Agnostic}
\item{My Day Job and my Mission in Life}
\end{itemize}
\end{frame}

\begin{frame}{Geeky Stories}
Talking About Code
\begin{itemize}
\item{Software Tutorials}
\item{User Guides}
\item{Code Blog}
\end{itemize}
Talking With Code
\begin{itemize}
\item{Research Reports}
\item{Analytics Emails}
\item{'Check Out This Awesome Graph' Blog Posts}
\end{itemize}
\end{frame}

\begin{frame}{In Common}
\begin{itemize}
\item{How do you know it's (still) right?}
\item{Leaving your Text Editor Sucks (WordPress, Microsoft Word)}
\end{itemize}
\end{frame}

\begin{frame}{What Dexy Does}
Write around live code, not about dead code.
\end{frame}

\begin{frame}[fragile]{Lets Have Some Code}

Let's play with the infochimps API. First, let's set up some constants:
\scriptsize
<< d['query-twitter-users.py|fn|idio|pycon|pyg|l']['base'] >>
\normalsize

\vspace{1cm}

Next, let's define a helper function:
\scriptsize
<< d['query-twitter-users.py|fn|idio|pycon|pyg|l']['fetch-chimp-data'] >>
\normalsize

\end{frame}

\begin{frame}[fragile]{Code}
Now we access the twitter influence metrics API.
We fetch data for a list of twitter users:

\scriptsize
<< d['query-twitter-users.py|fn|idio|pycon|pyg|l']['call'] >>
\normalsize
\end{frame}

\begin{frame}[fragile]{Code}
Here's what part of the data looks like:

\scriptsize
<< d['query-twitter-users.py|fn|idio|pycon|pyg|l']['pprint'] >>
\normalsize
\end{frame}

\begin{frame}[fragile]{Code}
We want to save this in a tabular format:

\scriptsize
<< d['query-twitter-users.py|fn|idio|pycon|pyg|l']['save-csv'] >>
\normalsize
\end{frame}

\begin{frame}[fragile]{Code}
Now we're going to switch to R to import and graph the data:

\scriptsize
<< d['graph-twitter-data.R|fn|idio|r|pyg|l']['import-csv'] >>
<< d['graph-twitter-data.R|fn|idio|r|pyg|l']['pairs'] >>
\normalsize
\end{frame}


\begin{frame}{So What?}
\begin{itemize}
\item{These slides and the handouts use the same source code.}
\item{The source code can be presented in different ways for different purposes.}
\item{I can change/fix the source code and it is automatically updated everywhere it is used.}
\item{I know that all the example scripts in my slides work, and that all quoted results are correct.}
\item{I easily and automatically shared data between scripts.}
\item{If I give you all my sources, you can reproduce my results.}
\end{itemize}
\end{frame}

\begin{frame}{What Else?}
\begin{itemize}
\item{Work in any language, or combination of languages.}
\item{Recycle your existing documentation (e.g. Javadocs) with extra awesome.}
\item{Easily write filters for any language that's not already supported, or to customize how they work.}
\item{Pull data from APIs (like Infochimps, Scraperwiki, Google Analytics), publish results to APIs (like WordPress, FigShare)}
\end{itemize}
\end{frame}

\begin{frame}{Writing}
Once you get proficient with Dexy, it very quickly becomes invisible.
Leaving you free to focus on communicating your work.
\end{frame}


\end{document}
